\section{\mitobo and ImageJ Plugins}
\label{chap:ImplPlugins}
The ImageJ plugin concept is a very powerful tool to extend the functionality of ImageJ
and, e.g., integrate third-party APIs. Accordingly, one of the main goals of the ImageJDev 
project developing ImageJ $2.0$ is to preserve the usability of available ImageJ plugins as far 
as possible. 

\mitobo seeks to extent ImageJ's functionality, 
i.e.~a natural aim is to have \mitobo operators directly available in ImageJ and ImageJ $2.0$ as
well, and to provide full integration in terms of easy interaction with other plugins. 
The straightforward solution for this goal would be to associate each single \mitobo operator with 
an explicit ImageJ $1$ or $2.0$ plugin. But, we have outlined the basic idea of generic operator 
execution implemented in \alida and \mitobo before. Consequently, as \mitobo's operator runners 
already provide functionality to execute all operators in a generic fashion, there is no need for 
explicit plugin implementations anymore. Rather it is sufficient to make the operator runners 
available as plugins in ImageJ and ImageJ $2.0$ -- which is already the case. 
Given that all operators can 
directly be used from within both ImageJ releases.

Indeed this approach of having \mitobo provide its own operator execution mechanisms also solves the
problem of compatibility to a certain degree. In particular, during the ongoing transmission period 
between both ImageJ versions it remains unclear of how to implement new plugin functionality that 
should at best be available simultaneously in ImageJ and ImageJ $2.0$. Although 
ImageJ $1$ plugins should in principal be supported by ImageJ $2$, both types of plugins are
not compatible with each other, and ImageJ $2$ plugins cannot be executed from within ImageJ $1$. 
Thus, by its operator runners being completely independent of ImageJ \mitobo offers an execution 
mechanism which is compatible with ImageJ and ImageJ $2.0$. 

Of course, this only holds if no ImageJ $1$ or ImageJ $2$ specific functionality is used by
the operators. As soon as this is done, an operator is tightly linked to one of the two ImageJ 
versions. While using ImageJ $1$ functionality might still allow to execute the operator in 
ImageJ $2$, the use of ImageJ $2$ renders the operator unsuitable for usage with ImageJ $1$.
However, this is not a \mitobo specific problem, but rather a question of software design.
Binding an implementation to an external library always renders the implementation useless without
that library. The only way to avoid such tight bindings with regard to operators and ImageJ is to
keep the functional core of operators free of ImageJ version specific dependencies as far as 
possible if the implementation targets at both releases.
 
