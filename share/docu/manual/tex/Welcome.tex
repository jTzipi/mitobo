\vspace*{-1cm}
\href{http://rsbweb.nih.gov/ij/}{ImageJ}\footnote{%
ImageJ -- {\em Image Processing and Analysis in Java},
\href{http://rsbweb.nih.gov/ij/}{http://rsbweb.nih.gov/ij/}}
is a widely-used Java toolkit for image processing and analysis. Particularly
in biomedical applications ImageJ has gained large
interest. ImageJ provides the user with a flexible graphical user
interface (GUI), with a large variety of basic built-in image processing operations,
and also with a huge collection of optional plugins downloadable from the
web. From a programmer's point of view, however, the ImageJ API provides less flexibility
to support plugin and application development. Especially easy data access
and exchange between more complex plugins below the GUI appear worth to be improved.

The ImageJ development project {\em ImageJDev }\footnote{ImageJDev website, 
\href{http://developer.imagej.net/}{http://developer.imagej.net/}}
is working towards a new release of ImageJ, {\em ImageJ $2.0$}, 
which is supposed to overcome many of these shortcomings of ImageJ on the programming level. 
The project developers are working on a complete restructuring of ImageJ's system architecture to 
decouple algorithms and user interfaces, and to allow for a much larger flexibility with regard to 
customizing ImageJ's core to specific needs\footnote{To differentiate between both versions of ImageJ throughout the remainder
of this guide, we will refer to the old release simply as 'ImageJ', and to the new version of the ImageJDev project as ImageJ $2.0$.}. 

\mitobo, which is the {\em {\bf \em M}icroscope {\bf \em I}mage Analysis {\bf \em To}ol{\bf \em Bo}x} 
developed at the Martin Luther University Halle-Wittenberg, targets at manifold goals. 
On the one hand it seeks to enhance the functionality and usability of ImageJ for programmers and users.
It shares the goals of an improved API and a strict decoupling of algorithmic 
functionality and user interfaces with the ImageJDev project. For example, \mitobo and ImageJ $2.0$ 
both define unified frameworks for implementing new functionality in terms of plugins and operators,
respectively, and both systems support the automatic generation of graphical user interfaces from 
these implementations.

On the other hand, however, \mitobo 
offers additional functionality going behind the features of ImageJ $2.0$.
With regard to graphical user interfaces it not just supports the automatic generation of user 
interfaces for single operators or plugins, respectively. 
Rather it extends the underlying model of image analysis procedures towards {\em workflows} 
where multiple operators are sequentially or in parallel applied to given input data for 
extracting desired results. 
This extension is inherently included in \mitobo's core and available to the user in 
terms of a graphical editor named {\bf \em Grappa} (Sec.~\ref{sec:grappa}) which supports users in designing workflows.
  
In general and contrary to the ImageJDev project \mitobo not only focusses on graphical user 
interfaces, but provides mature functionality for assessing implemented algorithms from commandline 
as well. In addition, it features a built-in mechanism for automatic documentation of analysis 
procedures which significantly simplifies the reconstruction of results and their long-term archival.
Finally, \mitobo not only inherently provides a comfortable framework for developing and 
implementing new image analysis algorithms. The toolbox also includes a selected collection of common image 
analysis techniques. They are mainly dedicated to the area of biomedical image analysis, but not restricted
to this field. Among these algorithms are for example segmentation approaches based on active 
contours for cell and scratch assay images, tools for neuron image analysis, or wavelet-based 
algorithms for detecting subcellular structures.

\mitobo is fully compatible with ImageJ, and features prototypical support for ImageJ $2.0$
which, however, will soon be completed towards full support. \mitobo's operators can be used both from 
within ImageJ and ImageJ $2.0$. In addition, \mitobo is capable of executing plugins from ImageJ 
$2.0$ and, e.g., include them in its process documentation. \mitobo is licensed under GPL, version
$3$, and can be downloaded from its website 
\href{http://www.informatik.uni-halle.de/mitobo}{http://www.informatik.uni-halle.de/mitobo}.
Please also have a look at the website for installation instructions.

\section{\mitobo's Core: Alida}
\mitobo's main paradigm is the idea of completely decoupling the implementation of functionality
from corresponding user interfaces for making this functionality available to users. The strict 
separation of algorithms and potential user interfaces offers largest flexibility to programmers.
It releases them from the cumbersome task to explicitly implement a variety of different interfaces.
At the same time the programmers are guided in implementing new functionality by clearly defined 
specifications and interfaces from which also users benefit. The usage of implemented 
functionality gets significantly simpler as the user interfaces for different algorithms follow 
common principles with regard to structure, appearance and handling.    

\mitobo\ is build on top of
\href{http://www.informatik.uni-halle.de/alida}{\alida}\hspace{-0.15cm}\footnotemark[4]
\footnotetext[4]{Alida website, \href{http://www.informatik.uni-halle.de/alida}{http://www.informatik.uni-halle.de/alida}}
from which it inherits the functionality for automatic user interface generation and automatic
documentation of analysis processes. \alida, which is the acronyme for 
{\em {\bf \em A}dvanced {\bf \em L}ibrary for {\bf \em I}ntegrated {\bf \em D}evelopment of 
Data Analysis {\bf \em A}pplications}, is a framework for supporting programmers in developing 
data analysis applications. It ships as a library providing core functionality for implementing
so-called {\em operators} as functional units for data analysis, and which also subsumes handy 
applications for running operators in a graphical or commandline context, as well as the graphical workflow editor {\bf \em Grappa}.

\alida basically interprets each data analysis pipeline as a 
process of modifying given input data (not necessarily image data) by a series of operations to produce the desired output data. 
Accordingly, all data analysis procedures in \alida\ are implemented by {\em operators} which are the only places to manipulate 
data. Furthermore, operators can be combined into {\em workflows}. 
Each workflow consists of a set of operators executed sequentially, in parallel or in a
nested fashion. Particularly the latter option naturally renders each workflow to form a new 
operator by itself, hence, in \alida a workflow is just an operator with extended functionality.  

It is straightforward to associate a selection of sequential and/or parallel operator calls 
with a directed graph data structure, where each operator is linked to a certain node of the graph, 
and input and output data is passed from one operator to another along the graph edges. This graph
together with the individual configuration parameters of all involved operators is a sufficient base
for lateron reconstructing all steps of data manipulation for a given result data object.
\alida\ allows to extract such {\em history graphs} explicitly and by this offers a powerful tool 
for the automatic documentation of data analysis processes. 
Note that these history graphs are not 
identical to the graphs implicitly defined by workflows as introduced above. 
Rather a workflow graph is often
only a subgraph of the history graph associated with the results generated by executing the 
workflow. This is due to the fact that operators usually rely on internal calls to other operators 
and these calls are not linked to an operator node in the workflow graph.   

\mitobo\ fully adopts the concepts of \alida, i.e.~\alida is an integral part of the \mitobo
framework. Consequently, image analysis procedures are realized in terms of operators which can be 
applied to data objects, e.g., images, regions or image primitives like keypoints and lines, 
and yield certain results. Contrary to \alida, which does not depend on any other data analysis
framework, \mitobo\ builds upon ImageJ. It basically relies on ImageJ's image data types, however,
does not interfere with ImageJ's plugin interfaces. Rather it allows to program ImageJ-compatible 
plugins based on the \alida\ operator concept.
Thus, in addition to the ImageJ interfaces which focus on plugin and script development,
\mitobo\ defines unique interfaces for the underlying image processing modules, i.e.~operators, in
terms of input/output data and parameters, and also with regard to the way how operators can be
invoked from other operators or user interfaces. This significantly improves data exchange and
operator handling, and establishes -- besides the already mentioned concept of automatic 
documentation -- a powerful fundament for adding new sophisticated features to ImageJ and ImageJ 
$2.0$. 

In the remainder of this guide we will outline the core functionality of \mitobo\ and its features with more 
detail. In doing so we will refer to the basic concepts of \alida only in a rudimentary fashion. 
If you are interested in more details about \alida, please refer to the \alida website\footnotemark[4] 
where you can find more information on \alida
bundled with examples of how to implement operators and how to execute them.

\section[\mitobo's Main Features: Operators, History Graphs and User Interfaces]
         {\mitobo's Main Features:\\ Operators, History Graphs and User Interfaces}
\label{sec:features}
\enlargethispage{0.2cm}
Besides providing modern image analysis tools and algorithms for microscope image analysis the
overall goal of the \mitobo\ project is to ease the analysis in terms of the development of appropriate algorithms and flexible user interfaces. These interfaces should, of
course, not only be designed for experts,
but also for researchers using image processing software as a tool rather than
developing their own
algorithms. \mitobo\ builds on top of ImageJ which has achieved large success and broad acceptance by researchers from
many different disciplines who need to solve their individual image analysis
problems. 

From the programmer point of view, ImageJ yields a suitable base for developing
image analysis tools in an integrated
framework. The programmer does not need to take care of, e.g., image display and zooming, as ImageJ
has answered such questions already. However, providing a certain degree of usability and easy integration
of new algorithms in terms of plugins is only one side of the medal. On the
other hand the underlying software
structures and interfaces should also provide a sufficient degree of comfort to the
programmer. In particular, image processing algorithms and user interfaces should be clearly separated from each other and data exchange between
different modules should be easy in terms of well-specified interfaces. 

As ImageJ is not optimally designed with regard to some of these aspects,
\mitobo\ does not exclusively focus on the
development of image analysis tools for microscope images, but also optimizes underlying software
in terms of data flow and data structures. In addition, self-documentation of image processing
pipelines is natively supported.

Based on the underlying concepts of \alida, \mitobo\ defines an image analysis pipeline as a sequence of operations sequentially or in
parallel applied to data that is handed over from operator to operator. Such a pipeline may be
viewed as a directed graph structure, where the nodes represent different operators and the data
flow is indicated by edges in between. From this interpretation of image analysis in
general, several design issues are derived that are embedded in the \alida\ core of
\mitobo. The concepts by themselves as well as functionality build on top of them are an integral part of \mitobo 
and provide users as well as programmers with enhanced image analysis tools and an improved
infrastructure for development and research.

\paragraph{Operator Concept.} The interpretation of image analysis pipelines as a
sequence of operations directly leads to the concept of {\em operators} implemented in \mitobo. 
All manipulations that are performed on data are done by operators. Vice versa operators
are the only actors that work on given data, modify the data or generate new
data entities from given input data. Accordingly, all image processing and
analysis algorithms in \mitobo\ are implemented in terms of operators with clearly specified input and output interfaces.

The operator concept is adopted from \alida. Technically \alida\ defines a common superclass for
data analysis modules denoted '{\tt ALDOperator}' from which the \mitobo\ main operator
class '{\tt MTBOperator}' is derived, and which all \mitobo\ operator classes extend. 
Furthermore, the concept incorporates a formal description of the interface of an operator,
i.e.~a formal specification of its inputs, outputs, and parameters, which is realized by the Java annotation
mechanism. In addition, there is only one possibility to invoke operators which is a single public routine to be
called from external which results in a unified invocation procedure for all operators. Among other things, this 
yields the base for automatic process documentation and user interface generation.
Moreover, every new operator implemented based on \mitobo and following these rules directly benefits from this
infrastructure, i.e.~can generically be executed and is automatically considered in extracting processing graphs.

\paragraph{Automatic Process Documentation and History Graphs.} The \alida\ operator concept with its transparent
interface specifications allows a unified handling of operators in various contexts, e.g., with
regard to graphical programming or automatic or semi-automatic code generation
where compatibility checks and operator calls have to be standardized. In
addition, the restriction of operator invocation to a single available method
also serves as a basis for another sophisticated feature of \mitobo\ which is the
fully automatic
documentation of image analysis pipelines. Given the interpretation of image
analysis pipelines as sequences of operators, all that remains to be done for
process documentation is to log the calls of all operators as well as their input and
output data and save their parameter settings. Together with
information about the order of operator
calls, for example represented in a directed graph data structure, these data
form a complete protocol of the pipeline and allow for longterm documentation of
analysis processes. In particular, linked to specific result data
objects, they allow to reproduce all results ever produced during the course of
algorithm development, testing, or
experimental evaluation.

The concept of automatic process documentation is realized in \alida, thus, it is also an integral part of
\mitobo\ and directly embedded into the operator concept. Operators provide internal functionality to
store process data during the course of an analysis procedure which later on can be extracted in terms of a
{\em history graph} file in XML format. Such a graph is associated with each data object being the
result of a certain operator or sequence of operations. Although the majority of objects will be
images, processing histories can also be associated with
segmentation results like regions or contours as well as histograms or any
other numerical data. See Figure~\ref{fig:DAG} for an example processing graph.

\paragraph{User Interfaces and Workflow Design} For (scientific) programmers the development of new image analysis algorithms
is mainly linked to the implementation of functionality to achieve desired results. For users who would like to 
benefit from new algorithms in their daily work the availability of suitable user interfaces is equally important. 
But as interface design and development usually coincides with a significantly increased amount of time for the programmer 
required to provide users with handy interfaces this aspect is often neglected. 

The unified invocation procedures for operators in \alida and their clear specification of input and output parameters pave 
the way for releasing programmers from the cumbersome task of implementing user interfaces. \alida, and consequently \mitobo,
inherently include mechanisms to automatically generate graphical as well as commandline user interfaces from operators in a 
completely generic fashion. Particularly, \alida's built-in support for a large variety of standard Java datatypes as operator 
parameters as well as for more sophisticated types like arrays, collections or enumerations renders it quite easy to implement
new functionality including suitable user interfaces with a minimum of programming overhead. 

And also users benefit from \mitobo's automatically generated user interfaces. Naturally all interfaces 
follow the same design and handling principles, e.g., all of them provide access to \mitobo's online help or allow for running 
the operator in batch mode. Moreover, \alida and \mitobo not only automatically generate user interfaces for single operators, 
but inherently support also the design of complete workflows formed by multiple operators which only in combination yield 
desired results. The graphical workflow editor {\bf \em Grappa} being shipped with \alida and \mitobo targets at supporting 
users in designing new workflows in a graphical manner. It integrates all operators implemented in the \mitobo framework as 
potential nodes of workflow graphs and allows for linking input and output parameters of these nodes to define more 
complex workflows. Thereby it takes full benefit of \alida's functionality, i.e.~Grappa offers automatically generated graphical
components for node configuration, data compatibility checks on dragging edges, and the fully automatic verification of operator 
configurations at runtime.

\section{About this Guide}
\label{sec:about}
This guide is organized in a short introductory section and two main parts. The introductory
section -- which you most probably have already read in the last few minutes -- is dealing with
some general remarks, while the two main parts provide more details. 
The first part is dedicated to users who are interested in using \mitobo\
plugins or commandline tools for their own work, e.g., to benefit from the automatic process documentation
capabilites of \mitobo or to adopt provided image analysis algorithms for solving their own image
processing problems. 
The second part introduces the reader to some more internals of \mitobo,
i.e.~it presents more details about the operator concept and how to use it with own
code, more information about specific \mitobo\ data types and also about programming with \mitobo\
in general. Note that only some few basics of the underlying concept of \alida\ are presented. 
For more details about its functionality and technical implementation refer
to the \alida\ manual to be found on the \alida\ webpage\footnotemark[4].

We hope that the new perspectives that \alida and in particular \mitobo\ open with 
their concepts might be helpful for developers and users of image processing applications, 
and by this extend ImageJ's selection of valuable features.  
If something remains to be clarified or if you have further notes and comments, just write an email
to us at {\tt mitobo@informatik.uni-halle.de}. We are happy to get in touch with you! 
