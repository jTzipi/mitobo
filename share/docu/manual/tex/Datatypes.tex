\chapter{\mitobo Data Types}
\label{chap:dataTypes}
\mitobo\ defines a set of its own data types. Besides new image data types
improving the ImageJ image classes, these
include for example regions and contours and some other data type primitives frequently used
with regard to image analysis applications. 
Most data types can be found in the package {\tt 'de.unihalle.informatik.MiToBo.core.datatypes'} and its
subpackages.
To allow for easy identification of the datatypes the
classnames of the data types in \mitobo\ always start with {\tt 'MTB'}, like in {\tt 'MTBRegion2D'} or {\tt
'MTBImageDouble'}.

There are several reasons why \mitobo\ implements its own data types and not
simply builds on top of the data types provided by ImageJ. First of all the handling of data objects in ImageJ is solved only in
a rudimentary fashion, at least with regard to the API. As there are only some few explicit data types apart from
images in ImageJ, data access or exchange is often cumbersome. Accordingly, \mitobo\ tries to enhance the usability and
flexibility of image processing modules by defining its own data types trying to overcome some limitations
nowadays present in ImageJ\footnote{Note that also ImageJ $2.0$ will provide a larger flexibility in data type handling, in particular 
more flexible image data types based on {\em ImgLib2}, website \href{http://fiji.sc/ImgLib2}{http://fiji.sc/ImgLib2},
will be available. Support for these extended image data types in \mitobo is planned.}.

Furthermore \mitobo defines some specific needs for data types with regard to its
feature of automatic process documentation (cf.~Sec.~\ref{sec:history}). 
Although \mitobo\ operators in principal support almost all available kinds of
objects as input and output parameters for operators, some few object
types cannot be handled natively within our concept of automatic process documentation.
Among those data types are for example Java's native data types like {\tt int} or {\tt double}, and wrapper classes like {\tt
'Integer'} and {\tt 'Double'}, i.e.~-- more generally speaking -- all
classes that implement the comparison of objects based on equality of object
values. If objects of these kinds should be used as operator input or output
parameters and, in particular, should be handed over from one operator to another, proper documentation of these data flows in the 
processing history can only be guaranteed by wrapping them in data types providing unique object identification independent of 
the current value. Accordingly, for some basic and frequently used data types \mitobo\ implements such data object wrappers.
They can be found in the package {\tt 'de.unihalle.informatik.MiToBo.core.datatypes.wrapper'}. 

Regarding automatic process documentation sometimes proper
documentation of operator configurations requires more than just logging an
input or output parameter's type and current value. There might be other object
parameters that are worth to be documented, e.g., like certain image-specific properties in case of
images. To support the documentation of such object properties \alida defines a basic data type class supporting management and
automated documentation of additional object properties. This class, {\tt ALDData}, 
serves as superclass for most \mitobo data types and can easily be adopted as basis for new data types. 

In the following sections first the properties of \alida's data type base class 
'{\tt ALDData}' will be outlined, and then we will discuss the different features and motivations 
of the most important \mitobo data types. 

\section{The Data Type Class {\tt ALDData} and its Properties}
\alida and \mitobo, respectively, allow to represent data and image processing pipelines as graph data structures, i.e.~history graphs. 
In particular, for each data object being the result of an analysis process composed of a series of data manipulations by \alida or
\mitobo operators,
the history graph allows to backtrace each single intermediate processing step subsuming all interactions with other
objects and the parameter settings of the involved operators. These data, together with the overall structure of the graph, already
draw a detailed picture of the process pipeline. However, sometimes extended information about manipulated and generated data
objects, i.e.~input and output parameters of the operators, are of interest that rise beyond the default data, like name, object
class and package. 

\alida defines the super class '{\tt de.unihalle.informatik.Alida.operator.ALDData}' to support
adding such specific information to input and output parameter objects of operators. The
class mainly adds the concept of data type {\em properties} to data types
derived from this class, allowing programmers to further characterize
objects in the processing history and also in general. Properties of
operator input and output parameter objects are automatically embedded in the history graph representation. 
Each time a data object passes an output port, i.e.~is taken out of an operator, 
the properties will be associated with the corresponding data port in the
graph. When later on viewing the graph with \mtbc (cf.~Sec.~\ref{sec:history}), the properties can then be
displayed as additional information of the corresponding ports. 

A property is basically given as a pair of key and value and is supposed to specify object characteristics. For
example in case of the \mitobo\ image data types properties subsume information like physical image and pixel sizes in all
dimensions and the units of the axes. Examplary key value pairs are shown in Table \ref{tab:propImg}.

For setting and getting object properties '{\tt ALDData}' defines two methods:
\begin{itemize}
    \item {\tt public void setProperty( String key, Object obj )}\\
    	allows to set a property named 'key' to the string representation of 'obj'
    \item {\tt public String getProperty( String key )}\\
		returns the string describing the value of the property named 'key'
\end{itemize}

Internally the properties are stored in a hashtable of the Java type {\tt 'Hashtable<String, String>'} to be found in
the package {\tt java.util}. Accordingly, keys and values are represented as strings. Nevertheless,
for convenience an arbitrary object can be handed over to the set routine as
shown above. It is automatically converted to a string via its {\tt toString()} method that consequently should return an informative description of the object at hand.
\begin{center}
\begin{table}[t]
\begin{center}
\begin{tabular}{l|l}
Property & Value \\
\hline
{\em location}  & ''/home/user/images/microscope.tif'' \\
{\em StepsizeX} & ''1''  \\
{\em StepsizeZ} & ''0.5''\\
{\em UnitX}     & ''cm'' \\
\ldots & \ldots 
\end{tabular}
\caption[Example properties of {\tt MTBImage}.]
  {\label{tab:propImg}Examplary properties and its values for an object of type {\tt MTBImage}.}
\end{center}
\end{table}
\end{center}

The programmer of a new \mitobo\ data type is in general allowed to choose arbitrary names for
the object properties without any restrictions, apart from one exception. There is one property predefined for all
\alida and \mitobo data types which is the property denoted '{\tt location}'. The location
of a data object defines the place of origin where the data object is coming from. This can be the place where it is physically stored, i.e.~the name of a
file on disk or an URL, or it can point to a virtual location if the object was generated by an operator in the course
of the processing pipeline. Note that although this property is by default
attached to all data types extending {\tt 'ALDData'}, it is, however, only set
automatically for \mitobo\ images by \mitobo's image I/O operators. For other data types setting the location
to proper values remains to the responsibility of the programmer of
the specific data type. 

To set and read the location of an object the following methods are available:
\begin{itemize}
 	\item {\tt public void setLocation( String location )}\\
 		sets the object location to the given string
    \item {\tt public String getLocation() }\\
 		returns the current location of the object 
\end{itemize} 

Note that there are no automatic checks to ensure that property names are unique. Thus, if the {\tt setProperty()} method
is called on a property which is already defined its previous value will be overwritten. This is
particularly true for the property '{\tt location}', so this key should never be
used by the programmer within another context than intended to omit confusion.


\section{Images in \mitobo: {\tt MTBImage}}
\label{sec:MTBImage}	
\mitobo defines its own image classes, namely \mtbimg~and its subclasses (which can all be found in the package
{\tt de.unihalle.informatik.MiToBo.datatypes.images}), for the following reasons:
\begin{itemize}
  \item extended pixel value precision to support all primitive numeric
  data types of Java
  \item easy access to image pixel data, but also to properties like physical pixel
  size etc.
  \item additional functionality for \mitobo's operator concept, i.e.~for documentation of
  specific image properties
\end{itemize}

In this section the internals of the \mitobo image data types are explained with more detail. The section is 
roughly divided into the following parts. At
first, some important details about the structure of \mtbimg~are given and available
image types are introduced. An overview of most common methods
for creation and manipulation of \mtbimg s follows. The section is closed
by the description of file I/O for \mtbimg~objects and how it integrates in \mitobo's
operator concept.

\subsection{The Ideas Behind \mtbimg}
\mtbimg~was not developed to fully replace ImageJ's 
\imgplus, but rather to wrap the \imgplus~objects if possible.
The most convenient way to create
a \mtbimg~from an existing \imgplus~object is to simply specify the \imgplus~as input parameter for the
method 
\centerline{\texttt{public static} \mtbimg{\tt .createMTBImage(ImagePlus img)}.}
The created \mtbimg~holds a reference to that \imgplus~object and explicitly stores the
image size as well as physical pixel size and units if available. For fast pixel
access, the \mtbimg~keeps direct references to the ImageJ data array or arrays in case
of a (hyper-)stack.

When a \mtbimg~is created from an \imgplus~the instantiated \mtbimg~must uniquely be
associated with the specified \imgplus, as no new \imgplus~is
created, but the existing one is used. This case occurs very often, e.g., when an
image window is selected from the ImageJ GUI and used as input for a \mitobo operator or plugin. 
Therefore an additional reference to the initially created \mtbimg~is added to the \textit{properties}
hashtable of the \imgplus. When an
\imgplus~with a reference to an existing \mtbimg~is passed to {\tt
createMTBImage(ImagePlus img)}, the existing \mtbimg~is simply returned.

Another aspect of \mtbimg~is to think of an \imgplus~as a 5-dimensional
image, which is the highest possible dimensionality of an image in ImageJ
(commonly denoted as 'hyperstack'). To provide easy access to higher dimensional image data
methods exist to access data in 5D hyperstacks, 3D stacks and 2D images,
which will be discussed with more detail in Section \ref{sssec:mtbimgfunc}.\\ 
\mtbimg~objects are designed in a similar way as ImageJ's \imgproc. You usually reference 
them by the abstract type \mtbimg, while one of its subclasses is actually instantiated.


\subsection{Subclasses of \mtbimg: Image Types}
One reason to develop a new image type was the limitation of ImageJ
images to 32-bit pixel value precision. The need for a 64-bit precision
floating-point image type to store results with higher accuracy was obvious.
Also the lack of a (true) 32-bit integer type in ImageJ can bear some
problems, e.g., when consecutive labels are given to image regions especially in
higher dimensional data. 

The concrete subclasses of {\tt MTBImage} indicate their types by their names. 
They share the common prefix '\mtbimg' followed by the name of the Java data type of their pixel values. 
The following list shows the image types available in \mitobo:
\begin{itemize}
  \item \mtbimg{\tt Byte} for {\tt byte}-type pixel values (unsigned as
  in ImageJ)
  \item \mtbimg{\tt Short} for {\tt short}-type pixel values (unsigned as in
  ImageJ)
  \item \mtbimg{\tt Int} for {\tt int}-type pixel values
  \item \mtbimg{\tt Float} for {\tt float}-type pixel values
  \item \mtbimg{\tt Double} for {\tt double}-type pixel values
  \item \mtbimg{\tt RGB} for three {\tt byte}-type pixel values, one
  for each color channel red, green and blue (unsigned)
\end{itemize}

All these image types share the same interface, but can be subdivided into two categories.
On the one hand there are image types that are directly linked to a corresponding ImageJ type and, thus,
simply wrap a corresponding \imgplus. On the other hander there are types that do not have a corresponding
ImageJ type. If only object references of type \mtbimg\ are used in an implementation, there is no difference 
between both classes. However, if functionality directly related to an \imgplus~object is requested or accessed 
(e.g., calls to ImageJ functions or the display of images in ImageJ's GUI), please keep in mind the differences
described in the following two paragraphs.

\paragraph{MTBImages with corresponding ImageJ types.}
If values of an \mtbimg~are changed which simply wraps the corresponding
\imgplus, the changes are directly applied to that \imgplus~as well, because 
\mtbimg~and \imgplus~share the same data arrays. Table \ref{tab:mtbimgIJ}
lists the subtypes of \mtbimg~of this category and their corresponding ImageJ image types.
\begin{table}[h]
	\begin{center}
	\begin{tabular}{|l|c|}
	\hline
	\mtbimg~subtype & \imgproc~of corresponding \imgplus \\
	\hline\hline
	\mtbimg{\tt Byte} & {\tt ByteProcessor} \\
	\hline
	\mtbimg{\tt Short} & {\tt ShortProcessor} \\
	\hline
	\mtbimg{\tt Float} & {\tt FloatProcessor} \\
	\hline
	\end{tabular}
	\end{center}
\caption{\mtbimg~types with corresponding ImageJ types.}
\label{tab:mtbimgIJ}
\end{table}

\paragraph{MTBImages without corresponding ImageJ types.}
\mtbimg s which cannot be represented by corresponding ImageJ types keep their
own data arrays and are not linked to an \imgplus~object upon creation.
Images of such data types cannot be instantiated by the {\tt createMTBImage(ImagePlus img)} method. These images are
usually constructed from scratch by specifying datatype and image size, or by
converting another \mtbimg~to that datatype.
Nevertheless, quite often an \imgplus~object needs to be associated to these images as well, usually for visualization
purposes.
\mtbimg~provides the function {\tt getImagePlus()} to obtain such an \imgplus. The
\imgplus~created is firmly associated with the \mtbimg. The new \imgplus~is of that ImageJ type which is supposed to provide the least
loss of information compared to the \mitobo source image. 
Note that \mtbimg~provides its own {\tt show()} and {\tt updateAndRepaint()} methods which already internally call the 
{\tt getImagePlus()} method. Consequently, it is not necessary to explicitly get an \imgplus~object for pure displaying purposes. \\[0.25cm]
\textbf{Important:\\
Always keep in mind, that a second image data object is kept in
memory, once {\tt getImagePlus()} or the displaying methods are called!}\\[0.25cm]
 Table \ref{tab:mtbimgNonIJ} describes the \mtbimg~types that do not have a
 corresponding ImageJ type and explains, how they are mapped to \imgplus.

\begin{table}[h]
	\begin{center}
	\begin{tabular}{|l|c|p{0.29\textwidth}|}
	\hline
	\mtbimg~subtype & \imgproc~of created \imgplus & Pixel value conversion \\ 
	\hline\hline
	\mtbimg{\tt Int} & {\tt FloatProcessor} & cast from {\tt int} to {\tt float} \\
	\hline
	\mtbimg{\tt Double} & {\tt FloatProcessor} & cast from {\tt double} to {\tt
	float} \\
	\hline
	\mtbimg{\tt RGB} & {\tt ColorProcessor} & lossless encoding of three {\tt
	byte} values to ImageJ's {\tt int} color representation \\
	\hline
	\end{tabular}
	\end{center}
\caption{\mtbimg~types without corresponding ImageJ types.}
\label{tab:mtbimgNonIJ}
\end{table}

\subsection{Construction, Data Access and Other Useful Methods of \mtbimg}
\label{sssec:mtbimgfunc} 
This subsection gives a short overview of the methods of \mtbimg~that are
widely used when working with \mitobo. A full description can be found in the
Javadoc \href{http://www.informatik.uni-halle.de/mitobo/api/index.html}{API} of \mitobo.\\[0.5cm]
At first, methods to create new \mtbimg s are presented. As there are no visible
constructors, you have to use the following {\tt static} factory functions:
\begin{itemize}
  \item {\tt public static \mtbimg~createMTBImage(\imgplus~img)}\\
  		creates a new \mtbimg~ of the correct subtype, which is uniquely linked to
  		the \imgplus
  \item {\tt public static \mtbimg~createMTBImage(int sizeX, int sizeY, int
  sizeZ, int sizeT, int sizeC, MTBImageType type)}\\
      	creates a new \mtbimg~ from scratch given the size and the data type of the
      	new image
\end{itemize}
The following methods can be used to create \mtbimg s from existing ones:
\begin{itemize}
  \item {\tt \mtbimg~duplicate()}\\
  		duplicates a \mtbimg.
  \item {\tt \mtbimg~convertType(MTBImageType type, boolean scaleDown)}\\
  		creates a \mtbimg~of the given type from the source image
\end{itemize}
There are more methods (e.g., to create a new \mtbimg~only from a part of an
existing image), please refer to the API for the other methods.

Methods for image pixel data access are declared by the \mtbimg{\tt
Manipulator} interface, which is implemented by the \mtbimg\ class. The behavior of data
access methods is similar to ImageJ's {\tt getPixel(\ldots)} and {\tt putPixel(\ldots)} methods, which
return or take a value of type {\tt 'int'} to cover 8-bit to 32-bit values. \mtbimg~provides
the same methods called {\tt getValueInt(\ldots)} and {\tt setValueInt(\ldots)}, with the only
difference that {\tt int}s are casted and not reinterpreted in case of
underlying floating point data types. Keep in mind that methods dealing with {\tt byte} types return and take values in the 
range of $[0,255]$, and methods dealing with {\tt short} types return and take values in the range of $[0,65535]$, like in ImageJ
as well. To cover floating point types additional methods
exist, which return or take {\tt double} values. These methods are called {\tt getValueDouble(\ldots)} and {\tt putValueDouble(\ldots)}. 
They define the safest way to go, if you cannot be sure which kind of images is passed to your method.

A word to (hyper-)stacks: ImageJ holds an array of 2D images, no matter if the
image is three-, four- or five-dimensional. 2D images (called 
\textit{slices} in the following) in this array (called \textit{stack}) are
referenced by $1$ to $N$, where $N$ is the total number of slices. \mtbimg~uses
indexing that every programmer is familiar with, starting from $0$ to $(N-1)$.

Below the methods to access pixel values in \mtbimg~are reviewed:
\begin{itemize}
  \item {\tt int getValueInt(int x, int y, int z, int t, int c)}\\
  		returns the pixel value at position (x,y,z,t,c) as {\tt int}
  \item {\tt void putValueInt(int x, int y, int z, int t, int c, int value)}\\
  		sets the pixel value at position (x,y,z,t,c) using an {\tt int} as input value
  \item {\tt double getValueDouble(int x, int y, int z, int t, int c)}\\
  		returns the pixel value at position (x,y,z,t,c) as {\tt double}
  \item {\tt void putValueDouble(int x, int y, int z, int t, int c, double
  value)}\\
  		sets the pixel value at position (x,y,z,t,c) using a {\tt double} as input value		
\end{itemize}

\mtbimg{\tt RGB} can be modified in the same way as color images in ImageJ, by
encoding the color values as an {\tt int} value (please refer to the ImageJ documentation for details)
and then passing that {\tt int} value to the {\tt putValueInt()} or {\tt putValueDouble()} method. 
In addition, \mtbimg{\tt RGB} further provides methods to get
and set values of the different color channels separately, or even get and work
on the \mtbimg{\tt Byte}s that represent the separate color channels.

For working with 2D images or 3D stacks there are equivalent methods that take
only 2D (i.e.~(x,y)) or 3D (i.e.~(x,y,z)) coordinates. You can also use these methods to access
certain slices (2D images) or z-stacks (3D images) of a (hyper-)stack. Therefore
you can set internal variables of \mtbimg~to specify a ``current'' slice or
z-stack with the following methods:
\begin{itemize}
  \item {\tt void setActualSliceCoords(int z, int t, int c)}\\
  		sets the coordinates of the ``current'' slice
  \item {\tt void setActualSliceIndex(int idx)}\\
  		sets the index of the ``current'' slice, i.e.~the index in the array of slices
  \item {\tt void setActualZStackCoords(int t, int c)}\\
  		sets the coordinates of the ``current'' z-stack (leaves ``current''
  		slice index unchanged)
\end{itemize}

The image data should be accessed by the above methods to develop algorithms for
generic image types. The data access methods are kept as fast as possible 
(e.g., subsume no further function calls). However, be aware that for this reason the
specified coordinates are not verified in any way. This means that running out of the
data array's bounds will cause an {\tt ArrayOutOfBoundsException} which is not caught by \mitobo.

For fast processing of higher dimensional images, you should also be aware of
how to iterate through the pixels. The usual ordering in \imgplus~hyperstacks 
is XYCZT, while \mitobo's interface
order is XYZTC. You should therefore iterate over the pixels of a \mtbimg~as
shown in the example below:\\[0.5cm]
\texttt{
\noindent\mtbimg~img = \mtbimg.createMTBImage(100, 100, 100, 100, 100,\\[-0.1cm]
						\hspace*{8.3cm}MTBImageType.MTB\_BYTE);\\[0.3cm]
for (int c = 0; c < img.getSizeC(); c++)\\[-0.1cm]
\hspace*{0.5cm}for (int t = 0; t < img.getSizeT(); t++)\\[-0.1cm] 
\hspace*{1.0cm}for (int z = 0; z < img.getSizeZ(); z++) \\[-0.1cm]
\hspace*{1.5cm}for (int y = 0; y < img.getSizeY(); y++) \\[-0.1cm]
\hspace*{2.0cm}for (int x = 0; x < img.getSizeX(); x++) \\[-0.1cm]
\hspace*{2.5cm}img.putValueInt(x,y,z,t,c,255);
}

If slicewise processing is possible, you can simply iterate over all slices,
which produces less lines of code and is the fastest way to access all
pixels:\\[0.3cm]
\texttt{
\noindent\mtbimg~img = \mtbimg.createMTBImage(100, 100, 100, 100, 100,\\[-0.1cm]
						\hspace*{8.3cm}MTBImageType.MTB\_BYTE);\\[0.3cm]
for (int i = 0; i < img.getSizeStack(); i++) \{ \\[-0.1cm]
\hspace*{0.5cm}img.setActualSliceIndex(i);\\[0.3cm]
\hspace*{0.5cm}for (int y = 0; y < img.getSizeY(); y++) \\[-0.1cm]
\hspace*{1.0cm}for (int x = 0; x < img.getSizeX(); x++) \\[-0.1cm]
\hspace*{1.5cm}img.putValueInt(x,y,255); \\[-0.1cm]
\}
}

\subsection{\mtbimg~I/O and the \mitobo~Operator Concept}
\mtbimg~extends the {\tt ALDData} base class for data types and therefore fully integrates in
\alida's and \mitobo's operator concept. 
A difference to other {\tt ALDData} types is, however, the
file input and output. \mtbimg~objects can be written to and read from disk
using the {\tt ImageWriterMTB} and {\tt ImageReaderMTB} operators, which can be
found in the package {\tt de.unihalle.informatik.MiToBo.io.images}.

The output of the save operator is comprised of two separate files: one image file in any supported
format, and a file (with ending .ald) that contains the image's processing history (see
Sec.~\ref{sec:history}). This history file -- if present -- is automatically
loaded when the image is opened using \mitobo's {\tt ImageReaderMTB} operator. The processing
history files can be examined using {\tt chipory}.

\mitobo~relies on the Bio-Formats Library
(\url{http://www.loci.wisc.edu/software/bio-formats}) and, thus, allows reading and writing
all file formats supported by Bio-Formats. Bio-Formats is a sophisticated library for image I/O
targetting at the various file formats in biomedical imaging and being compatible with the Open Microscopy Environment
(OME) standard (\url{http://www.openmicroscopy.org}).


