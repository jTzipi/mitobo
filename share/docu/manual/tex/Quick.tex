Getting in touch with \mitobo\ is quite easy!\\
If \mitobo\ was successfully installed (cf.~Chap.~\ref{sec:install}), you
can use the included plugins and tools in the same way as you are familiar with 
from ImageJ. And of course feel free to use \mitobo\ for your own data types,
operators and plugin development to solve special problems of interest in the
field of image analysis and processing.\\
\subsection*{\underline{Simply use \mitobo\ in ImageJ}}
   \begin{enumerate}
    \item use \mitobo\ like every other ImageJ software tool or plugin
    \item \textbf{open image(s)} of interest (internal \mitobo\ IO-plugins are
          recommended)
    \item \textbf{choose a plugin} or tool out of \mitobo\
          (cf.~Sec.~\ref{sec:plugAndOp})
    \item \textbf{display} or process the plugin \textbf{results} in any way you
          can imagine
    \item \textbf{save} the plugin \textbf{results and} store the containing
          \textbf{history graph}
    \item \textbf{use the history for further research}, improvement,
          parameter\\tuning and so on (cf.~Chap.~\ref{chap:features})\\
\end{enumerate}

\subsection*{\underline{Write your own \mitobo\ operator}}
\begin{enumerate}
 \item \textbf{implement the interface} of the operator
 \begin{enumerate}
  \item implement a new operator (cf.~Sec.~\ref{sec:implOperators}) by
        \textbf{extending} the \textbf{abstract class MTBOperator}
  %%\item \textbf{implement a standard constructor}  without arguments, this
        %%constructor needs to do nothing, but needs to be implemented for
        %%self-documentation and code generation capabilities
  \item \textbf{define the operator descriptors} for all parameters,
        inputs, outputs, and supplemental arguments (not stored in the
        processing history)
  %%\item \textbf{input and output} arguments are \textbf{derived from} the
        %%abstract class \textbf{MTBData}
 \end{enumerate}
\item \textbf{implement the operation} of the operator
 \begin{enumerate}
  \item \textbf{overwrite the \lstinline+operate()+ method} which implements
	the operator \textbf{algorithm} per se
	%%and set the
        %%parameters, inputs, outputs, and supplemental arguments
  %%\item \textbf{implement} the operator \textbf{algorithm} per se\\
  \item optionally \textbf{implement getter and setter methods} for 
        parameters, inputs, outputs, and supplemental arguments
 \end{enumerate}
\end{enumerate}

\subsection*{\underline{Write your own \mitobo\ data type}}
 \begin{enumerate}
  \item \textbf{implement} a new \mitobo\ data type
        (cf.~Sec.~\ref{sec:dataTypes}) \textbf{as subclass from} the class
        \textbf{MTBData}
  \item \textbf{set the predefined ’location’ property}, where it is physically
        stored\\
        (file name on disc, URL, point to virtual location, et cetera)
  \item \textbf{set additional properties} of the data type\\
   (feel free to choose arbitrary names for the data type properties)
 \end{enumerate}

\subsection*{\underline{Write your own \mitobo\ plugin}}
 \begin{enumerate}
  \item \textbf{implement your plugin} as
  \begin{enumerate}
    \item standard ImageJ plugin (cf.~Sec.~\ref{sec:PlugIn})
    \item \mitobo\ operator plugin (cf.~Sec.~\ref{sec:MTBPlugIn})
  \end{enumerate}
  \item simply \textbf{use operator(s)} (cf.~Sec.~\ref{sec:programmer}) 
        as included or implemented by yourself
  \item make \textbf{use} of \textbf{different kinds of} pre-defined
        (cf.~Sec.~\ref{sec:basicDataTypes}) or self-constructed \textbf{data
types},        derived from the abstract class \lstinline+MTBData+
  \item \textbf{set the output} of your plugin
  \begin{enumerate}
   \item image, ROI, table and so on
   \item any kind of data types and their containing history graph\\
  \end{enumerate}
 \end{enumerate}
Using \mitobo\ you are always able to get self-documentation of
your image analyzing or processing work flow. This concept is very helpful to
reproduce the data creation, just as the comparison of your parameter settings
and theirs subsequent results (cf.~Chap.~\ref{}).\\
\TODO{hier vielleicht besser eine Referenz auf ein detailiertes Beispiel, wo man
auch mal die Parameter im Graph sieht oder aehnliches ?}
