When using ImageJ or ImageJ $2.0$ it is quite natural to interact with plugins and operators, respectively, via 
graphical user interfaces. Nevertheless, quite often not only some few images need to be analyzed,
but nowadays even high-throughput processing is an important issue. While ImageJ has built-in
functionality for scripting and macros, \mitobo in addition offers a handy commandline tool by 
which all of its operators (and also ImageJ $2.0$ plugins) can directly be called from console in a
generic fashion. This way they can easily be applied to large collections of image data and also be used
from within shell scripts or comparable frameworks.  

The commandline operator runner is essentially an \alida tool, i.e.~is to be found in the package
{\tt de.unihalle.informatik.Alida.tools.OpRunner}. Its basic usage is as follows:
{\small
\begin{center}
{\tt java  de.unihalle.informatik.Alida.tools.OpRunner  <operator name>  \{
parameter=value\}}
\end{center}
}
Its first argument is the name of the operator class to be executed. You do not need to specify its complete 
package name, but the simple class name by itself is usually sufficient. Moreoever, the commandline 
operator runner also supports auto-completion if the given prefix is unique among all operators 
and plugins found on the classpath.  

Following the operator name the commandline operator runner expects a set of 'name=value' pairs 
for the parameters of the operator. While the parameter names are defined by the operator's 
member variables (execute the operator runner with option '-n' to only print its parameters), 
the exact 
syntax of the value strings depends on the data types of the parameter and their data I/O providers. 
For native data types and 
strings it is sufficient to simply provide the concrete values, for image data types a filename
is required from where to load the image. The following example illustrates this by calling 
an operator for image erosion:
{\small
\begin{center}
{\tt java de.unihalle.informatik.Alida.tools.OpRunner ImgErode inImg=test.tif
masksize=3}
\end{center}
}
However, the commandline operator runner and its built-in parser also support far more complex 
calls to operators. To illustrate this, below the call to an extended snake segmentation operator
is shown. The operator {\tt 'SnakeOptimizerCoupled'} allows to apply multiple simple snake 
optimizer operators of type 
{\tt 'SnakeOptimizerSingleVarCalc'} simultaneously to one image, given by the parameter 'inImg'. 
The operator among others takes a prototypical operator object 
of type {\tt 'SnakeOptimizerSingleVarCalc'} as input parameter ('snakeOptimizer'). This object 
by itself expects, e.g., a weighted set of energies ('energySet') which is formed by a collection 
of energy objects ('energies') and an array of corresponding weights ('weights'). The energy
objects can again be parametrized, e.g., in the example below the energy object of type 
{\tt 'MTBSnakeEnergyCD\_CVRegionFit'} defines two parameters {\tt lambda\_in} and {\tt lambda\_out}:
\begin{center}
\hspace*{-1cm}{\tt java  de.unihalle.informatik.Alida.tools.OpRunner 
SnakeOptimizerCoupled $\backslash$ }\\
{\tt  initialSnakes=RoiSet.xml inImg=cell.tif outSnakes=snakesOut.xml $\backslash$}\\
\hspace*{-1.25cm}{\tt snakeOptimizer='\$SnakeOptimizerSingleVarCalc:\{energySet= $\backslash$}\\
{\tt \{energies=[\$MTBSnakeEnergyCD\_CVRegionFit:\{lambda\_in=1.0,$\backslash$}\\
\hspace*{9cm}{\tt lambda\_out=5.0\}],$\backslash$}\\
\hspace*{-7.5cm}{\tt weights=[1.0]\}\}'}
\end{center}

For accessing the results of an operator invoked by the commandline runner it is required to 
specify targets for the operator's output parameters. In case of the image erosion operator it provides
its result through an output parameter denoted 'resultImg'. Consequently, to save the eroded image to 
file it is sufficient to extend the operator call as follows: 
\begin{center}
{\tt java  de.unihalle.informatik.Alida.tools.OpRunner ImgErode inImg=test.tif
$\backslash$\\
\hspace*{5cm} masksize=3 resultImg=result.tif}
\end{center}
Note that output parameters for which no target is provided will be ignored by the commandline 
operator runner, thus, are not available upon termination.

The examples shown above only provide you with a very brief overview of the functionality of the 
commandline operator runner. To learn more about all its options and features, please refer to 
the documentation of \alida where more details can be found.  